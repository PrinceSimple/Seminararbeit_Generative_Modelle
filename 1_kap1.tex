\section{Einführung in Machine Learning}\label{Machine Learning}
Machine Learning (ML) ist ein interdisziplinäres Teilgebiet der künstlichen Intelligenz, welches die Generierung von »Wissen« aus »Erfahrung« bezweckt. Mithilfe von ML werden IT-Systeme in die Lage versetzt, auf Basis vorhandener Datenbestände und Algorithmen Muster und Gesetzmäßigkeiten zu erkennen und ein komplexes Modell der Daten zu entwickeln. Das Modell, und damit die automatisch erworbene Wissensrepräsentation, kann dann verallgemeinert für neue Problemlösungen oder für die Analyse von bisher unbekannten Daten verwendet werden. Dazu verwendet ML zahlreiche Methoden, wobei in vorliegender Seminararbeit nur jene erläutert werden, welche für das allgemeine Verständnis generativer Modelle nötig sind.
%\cite{mm2009}

\subsection{Deep Learning}\label{Deep Learning}
%$p_{model}$, $p_{model}$ $\boldsymbol{\hat{\theta}}$
%\textsl{Lorem} 
Deep Learning umfasst eine Reihe von Algorithmen, deren Architektur auf mehreren nacheinander angeordneten Verarbeitungsschichten basiert und deren Aufgabe es ist, aus unstrukturierten Daten übergeordnete Repräsentationen zu lernen.
Deep-Learning-Modelle können auf strukturierte Daten angewendet werden, aber ihre eigentliche Stärke, insbesondere im Hinblick auf die generative Modellierung, ergibt sich aus ihrer Fähigkeit, unstrukturierte Daten verarbeiten zu können. In den meisten Fällen liegt es in der Natur der Aufgabe, unstrukturierte Daten wie neue Bilder oder echte Textzeilen zu erzeugen. Daher verwundert es wenig, dass vor allem die Entwicklungen im Deep Learning einen bedeutenden Einfluss auf die generative Modellierung hatten\cite{fos19}.
Bei der Mehrheit aller Deep-Learning-Modelle handelt es sich um künstliche neuronale Netze (KNNs, kurz »neuronale Netze«) mit mehreren hintereinanderliegenden verborgenen Schichten. Aus diesem Grund ist Deep Learning inzwischen fast zum Synonym für tiefe neuronale Netzwerke geworden. Es ist jedoch wichtig darauf hinzuweisen, dass jedes System, das mehrere Schichten verwendet, um übergeordnete Darstellungen der Eingangsdaten zu lernen, auch eine Form des Deep Learning ist (z. B. Deep-Believe-Netzwerke und Deep-Boltzmann-Maschinen).
Um Annahmen nicht bereits im Voraus treffen zu müssen, wird ein Modell benötigt, das die relevanten Strukturen aus den Daten selbst ableiten kann. Hierbei zeichnet sich Deep Learning aus. Die Tatsache, dass Deep Learning seine eigenen Merkmale in einem niederdimensionalen Raum finden kann, bedeutet eine Form des \hyperref[Representation Learning]{Representation Learnings}\cite{goodl16}.

\subsection{überwachtes und unüberwachtes Lernen}\label{überwachtes / unüberwachtes Lernen}
Wenn ein Modell anhand gekennzeichneter Trainingsdaten (\emph{labeled data}) lernen soll, handelt es sich um überwachtes Lernen. Die Methode richtet sich also nach einer im Vorhinein festgelegten Ausgabe, deren Ergebnisse bekannt sind. Die Ergebnisse des Lernprozesses können mit den bekannten, richtigen Ergebnissen verglichen, also überwacht, werden. Die Überwachung bezieht sich dabei nur auf die sogenannten Trainingsdaten, mit denen ein KNN für die Lösung einer Aufgabe trainiert wird. Im produktiven Einsatz des Modells wird grundsätzlich nicht überwacht.
Zu den überwachten Lernverfahren zählen alle Verfahren zur Klassifikation oder Regression, beispielsweise mit Algorithmen wie k-nearest-Neighbour, Random Forest, Support Vector Machines oder auch Verfahren der Dimensionsreduktion wie die lineare Diskriminanzanalyse. Der Nachteil von überwachtem Lernen besteht jedoch in einem oft sehr hohen manuellen Aufwand bei der Aufbereitung der Trainingsdaten.
Eine besondere Form des überwachten Lernens ist die des bestärkenden Lernens. Bestärkendes Lernen kommt dann zum Einsatz, wenn ein Endergebnis noch nicht bestimmbar ist, jedoch der Trend hin zum Erfolg oder Misserfolg erkennbar wird. In der Trainingsphase werden beim bestärkenden Lernen die korrekten Ergebnisse also nicht zur Verfügung gestellt, jedoch wird jedes Ergebnis bewertet, ob dieses (wahrscheinlich) in die richtige oder falsche Richtung geht.\cite{ras18}

Im Gegensatz zum überwachten Lernen werden beim unüberwachten Lernen weder gekennzeichnete, noch klassifizierte Trainingsdaten verwendet. Ziel dieses Ansatzes ist es, aus den Daten unbekannte Muster zu erkennen und Regeln aus diesen abzuleiten. Das ML-System nutzt Algorithmen, die die Struktur der Eingabedaten erkunden und daraus sinnvolle Informationen in den Ausgabedaten abbilden. Die Vorteile des unüberwachten Lernens bestehen in der teilweise vollautomatisierten Erstellung von Modellen. Dabei können diese eine sehr gute Prognose über neue Daten hervorbringen oder gar neue Inhalte erstellen (generative Modelle). Das Modell lernt mit jedem neuen Datensatz dazu und verfeinert gleichzeitig seine Berechnungen und Klassifizierungen. Ein manueller Eingriff ist dadurch nicht mehr notwendig. KNNs basieren auf diesem selbstlernenden Verfahren.

\subsection{Representation Learning}\label{Representation Learning}
Die Grundidee beim Lernen von Repräsentationen besteht darin, nicht den hochdimensionalen Stichprobenraum direkt zu modellieren, sondern jede Beobachtung im Trainingsdatensatz mittels eines niederdimensionalen latenten Raums zu erfassen. Das Modell muss dann eine Mapping-Funktion lernen, die einen Punkt im latenten Raum nehmen und ihn auf einen Punkt in dem ursprünglichen Raum abbilden kann. Mit anderen Worten: Jeder Punkt im latenten Raum ist die Darstellung einer hochdimensionalen Repräsentation der Eingabedaten, beispielsweise eines Bilds.



%\subsection{überwachtes und unüberwachtes Lernen}\label{überwachtes und unüberwachtes Lernen}
%Hier steht die Einleitung der Arbeit... Lorem ipsum dolor sit amet, consetetur sadipscing elitr, sed diam nonumy eirmod tempor invidunt ut labore et dolore magna aliquyam erat, sed diam voluptua. At vero eos et accusam et justo duo dolores et ea rebum. Stet clita kasd gubergren, no sea takimata sanctus est Lorem ipsum dolor sit amet. Lorem ipsum dolor sit amet, consetetur sadipscing elitr, sed diam nonumy eirmod tempor invidunt ut labore et dolore magna aliquyam erat, sed diam voluptua. At vero eos et accusam et justo duo dolores et ea rebum. Stet clita kasd gubergren, no sea takimata sanctus est Lorem ipsum dolor sit amet.