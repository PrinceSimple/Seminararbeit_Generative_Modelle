\section{Anwendungsmöglichkeiten generativer Modelle}\label{Anwendungsmöglichkeiten generativer Modelle}

Im Gegensatz zu diskriminativen Modellen bieten generative Modelle kaum Lösungsansätze für die Automatisierung von Validierungs- und Überwachungsprozessen in der Industrie. Generative Modelle eignen sich hingegen vor allem für die Erstellung von Inhalten wie Bilder, Texte oder Musik \cite[S.7]{fos19}.
Deshalb liegt das größte Potenzial der Entwicklung in der Kreativbranche. Schon heute zeigt sich der Trend zur prozeduralen Arbeitsweise in professionellen Softwareprodukten für die Erstellung von Inhalten, wie Substance Designer, Cinema4D und Houdini. Dort werden Texturen, Shader und ganze 3D-Modelle durchgehend parametrisiert erstellt und sind zur Laufzeit variabel. Wenn generative Deep-Learning-Modelle auf diese Parameter angewendet werden, lassen sich per Knopfdruck einfach Variationen und Abstraktionen erstellen, welche sich qualitativ nicht von den von Menschen gestalteten Entwürfen unterscheiden \cite{bcon18}. Der Trend zu \frqq{user generated content\flqq} begünstigt dazu den Datenwachstum und bereichert somit die Datenbasis zur Erforschung neuer generativer Methoden.

\subsection{Bild}
Bei der Erläuterung des CycleGAN wurde bereits erwähnt, dass der Stiltransfer ein bedeutender Punkt in der Forschung ist, wenn es um generative Modelle geht \cite{san18}. Zahlreiche mobile Applikationen setzen generatives Deep-Learning ein, um Filter zu entwickeln, die aus den Benutzeraufnahmen - teilweise in Echtzeit - kunstvolle Effekte durch Stiltransfer erzeugen \cite{hua17}. In diesen Apps werden auch Kombinationen aus diskriminativen Modellen und generativen Modellen benutzt, um Gesichter einerseits zu erkennen und andererseits beispielsweise altern zu lassen oder das Geschlecht zu ändern. Sogenannte \frqq{Deepfakes\flqq} bedienen sich dieser Architekturen, um Gesichter von Personen in Videos auszutauschen. Dieselben Methoden werden in der Filmproduktion eingesetzt, um fehlerhaft gedrehte Szenen nicht erneut drehen zu müssen. Auch wenn Webseiten wie \href{https://thispersondoesnotexist.com}{thispersondoesnotexist.com} eher als Machbarkeitsstudien für realistische Gesichtsfotos zu sehen sind, ist das Wirkpotenzial realistisch wirkender Bilder von Objekten für die Medienwirtschaft enorm.

Beim Rendern von physikalisch korrekten Lichtsituationen entsteht unerwünschtes Bildrauschen, welches nur vermieden werden kann, wenn die Samplerate des Renderers erhöht wird. Damit werden aber auch die Renderzeiten verlängert, was im Endeffekt immer eine Kostenabwägung ist. Mit Technologien wie Intels \emph{Open Image Denoise} oder \emph{Optix AI-Accelerated Denoiser} von Nvidia, welche auf dem Prinzip eines Autoencoders beruhen, kann man mittlerweile sehr einfach selbst stark verrauschte Bilder in ein akzeptables Endprodukt verwandeln und dadurch Kosten sparen \cite{bcon18}. SRGANs (Super Resolution Generative Adversarial Network) sind eine weitere Klasse von GANs und können niedrigaufgelöste Bilder in hochauflösende Bilder umwandeln. Dadurch können stark vergrößerte Bildausschnitte oder qualitativ minderwertiges Bildmaterial rekonstruiert werden und neue Erkenntnisse in Bildgebungsverfahrensforschung geben \cite{dblp16}.

\subsection{Audio}
Die meisten gezeigten Beispiele waren bisher von der Verarbeitung von Bilddaten geprägt. Neben der Erzeugung optisch ansprechender und aussagekräftiger Bilder finden generative Modelle jedoch auch praktische Anwendung im Bereich der Erzeugung von Musik, insbesondere bei Videospielen und Filmen. MuseNet von OpenAI ist in der Lage, unbegrenzt Musik in einem bestimmten Stil zu erzeugen und Genres zu neuen Kompositionen zu mischen. Mit der Magenta API von Google wurden leistungsstarke Werkzeuge geschaffen, die die Arbeitsweise von Künstlern erweitern, indem Kompositionen anhand von Parametern erstellt und verändert werden können \cite{rob19}. Die Aufgabenstellungen generativer Modelle können im Audiobereich in zwei wesentliche Felder aufgeteilt werden. Zum einen werden MIDI-Daten erzeugt, welche dann virtuelle Instrumente kontrollieren, um wohlklingende Kompositionen zu erhalten. Zum anderen besteht die Aufgabe in der Erzeugung digitaler Rohdaten, welche aus Zufallsrauschen synthetisiert werden, ähnlich dem StyleGAN. Darüber hinaus kommen Stiltransfermethoden zum Einsatz, um Sprecher auszutauschen oder die Stimme eines \emph{Text-To-Speech-Systems} den Nutzervorlieben anzupassen.

\subsection{Text}
Maschinengenerierte Texte sind eine besondere Herausforderung für generative Modelle, da semantische Hürden überwunden werden müssen, damit sie authentisch wirken. Ein wichtiger Anwendungsfall für automatisierte Texterstellung ist journalistischer Natur. Live-Berichterstattung, Wetterbericht, Newsticker und Spielberichte aus dem Massensport bedienen sich bereits seit längerem maschinengenerierter Texte. Auch im Content Marketing kommen generative Modelle zum Einsatz. In einfachen Textformen wie Bildunterschriften und Produkttexten können diese mit einer guten Datenbasis bereits überzeugende Texte liefern.

GPT-2 von OpenAI ist ein generatives Modell, welches insbesondere für Aufgaben im Rahmen der Erzeugung von sinnvollen Sätzen ausgelegt ist. Aufgrund von Bedenken, dass dieses Modell von böswilligen Dritten missbraucht werden könnte, z. B. um gefälschte Nachrichten, gefälschte Aufsätze, gefälschte Konten in den sozialen Medien oder Online-Imitationen von Personen zu erzeugen, hat OpenAI beschlossen, den Datensatz, den Quellcode und die Modellgewichte von GPT-2 nicht zu veröffentlichen \cite{fos19}.