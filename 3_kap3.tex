\section{Anwendungsmöglichkeiten generativer Modelle}\label{Anwendungsmöglichkeiten generativer Modelle}

Im Gegensatz zu diskriminativen Modellen bieten generative Modelle kaum Lösungansätze für die Automatisierung von Validierungs- und Überwachungsprozessen in der Industrie. Generative Modelle eignen sich hingegen vor allem für die Erstellung von Inhalten wie Bilder, Texte oder Musik.
Deshalb liegt das größte Potenzial der Entwicklung in der Kreativbranche.

Bild
- Stiltransfer (aiportraits.com)
- Entrauschen ( Nvidia denoiser )
- SRGAN
- SyleGAN2 (thispersondoesnotexist.com)

Ton
-Stiltranfer
Text