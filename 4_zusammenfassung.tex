\section{Zusammenfassung}\label{Zusammenfassung}

In dieser Seminararbeit wurde ein Überblick über die Methoden und Anwendungen von generativen Modellen in Machine Learning gegeben. Es sei darauf hingewiesen, dass dieser Überblick in keiner Weise einen Anspruch der Vollständigkeit hat. Machine Learning ist ein riesiges, interdisziplinäres Fachgebiet und wird ständig erweitert.
Generative Modelle und ihre Anwendungen stellen nur einen kleinen Teil dieses Fachgebietes dar und sind trotzdem nicht im Umfang dieser Seminararbeit vollständig zu erfassen. Der Leser sollte aber trotzdem einen Einblick in die Funktionsweise von generativen Deep Learning bekommen haben und eine Vorstellung davon bekommen haben, zu welchen Leistungen generative Modelle in der Lage sind. Es sollte klar geworden sein, wie durch kleine Veränderungen an einem generativen System immer bessere Ergebnisse erzielt werden können. Bei der Beschreibung der beiden Beispielmodelle wurde auf die Verwendung mathematischer Notation und die Erläuterung von tiefgehenden, wahrscheinlichkeitstheoretischen Ansätzen weitgehend verzichtet.
%%Im ersten Teil wurde der Leser in Machine Learning eingeführt und grundsätzliche Begriffe erläutert, die zur Einordnung generativer Modelle nötig sind. Die Grundlagen des Aufbaus und der Funktionsweise zweier Beispiele für die generative Verwendung im Deep Learning wurden beschrieben. Dabei wurde auf die Verwendung mathematischer Notation und die Erläuterung von tiefgehenden, wahrscheinlichkeitstheoretischen Ansätzen weitgehend verzichtet. Im dritten Teil wurden Anwendungsbeispiele generativer Modelle in verschiedenen Medienformen erläutert und das Potenzial für die Entwicklung in der Kreativbranche umrissen. Daraufhin lohnt es sich einen Ausblick zu wagen.

