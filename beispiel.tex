% Beispiel für Bild mit Fußnote
\begin{figure}[htb]
 \centering
 \includegraphics[width=0.4\textwidth,angle=45]{abb/logo1}
 \caption[Beispiel einer Bildbeschreibung]{Beispiel einer Bildbeschreibung\footnotemark}
\label{fig:beispiel1}
\end{figure}
\footnotetext{Bildquelle: Beispiel einer Bildquelle}

% Beispiel für Bildintegration
\begin{figure}[htb]
 \centering
 \includegraphics[width=0.3\textwidth,angle=0]{abb/logo2}
 \caption[Beschreibung]{Beschreibung}
\label{fig:Beschreibung}
\end{figure}

% Beispiel: Referenz auf Abbildung
Abbildung~\ref{fig:Beschreibung} [S.\pageref{fig:Beschreibung}]

% Beispiel: Tabelle 
\begin{center}
  \begin{tabular}{ | l | c | }
    \hline
    Überschrift 1 & Überschrift 2 \\ \hline \hline
    Info 1 & Info 2 \\ \hline
    Info 3 & Info 4 \\ \hline
    \hline
    \multicolumn{2}{|c|}{Info in einer Zelle} \\
    \hline
  \end{tabular}
\end{center}




% Beispiel für Formeln
Die Zuordnung aller möglichen Werte, welche eine Zufallsvariable annehmen kann nennt man \emph{Verteilungsfunktion} von $X$.

\begin{quotation}
Die Funktion F: $\mathbb{R} \rightarrow$ [0,1] mit $F(t) = P (X \le t)$ heißt Verteilungsfunktion von $X$.\footnote{Mustermann, vgl.~\cite{gfnips16}~[S.4]}
\end{quotation}

\begin{quotation}
Für eine stetige Zufallsvariable $X: \Omega \rightarrow \mathbb{R}$ heißt eine integrierbare, nichtnegative reelle Funktion $w: \mathbb{R} \rightarrow \mathbb{R}$ mit $F(x) = P(X \le x) = \int_{-\infty}^{x} w(t)dt$ die \emph{Dichte} oder \emph{Wahrscheinlichkeitsdichte} der Zufallsvariablen $X$.\footnote{Goodfellow, vgl.~\cite{gfg14}~[S.3]}
\end{quotation}
