\section{Ausblick}\label{ausblick}

Durch die rasante Entwicklung im Forschungsfeld generativer Modelle wurde in den letzten Jahren die Grenze der lösbaren Aufgaben immer weiter verschoben. Es lässt sich bereits heute erkennen, welche Schlüsselrolle die Methoden der generativen Modellierung in der digitalen Medienproduktion einnehmen werden. Dies wird im Gestaltungsprozess dazu führen, dass Medienproduzenten allein aus Gründen der Kosteneffizienz auf generative Modelle zurückgreifen werden. Den Designer bzw. Künstler werden diese Systeme auch auf längere Zeit jedoch nicht ersetzen können \cite{bcon18}. Sie werden eher den kreativen Prozess unterstützen und Ideevariationen erstellen, aus denen der Produzent dann auswählen kann. Mit {Text-To-Image}-Verfahren ist es bereits heute möglich aus rein deskriptivem Text Bilder von Objekten und Landschaften zu erzeugen \cite{zha17}.

Darüber hinaus ist eine Entwicklung hin zu personalisierten Medienprodukten wahrscheinlich. So könnten beispielsweise Radiosender für einzelne Personen entstehen, die je nach Situation die Hörpräferenzen des Nutzers bedienen. Die Handlung in Romanen könnte individualisiert verlaufen und Filme mit den bevorzugten Schauspielern ausgestattet werden \cite[S.283]{fos19}.
Dadurch wird sich aber auch die rechtliche Frage der Lizenzierung von maschinengenerierten Inhalten stellen und wie Medienprodukte als geistiges Eigentum geschützt werden können. Dieser Aspekt wird sicherlich in Zukunft noch häufig zu gesellschaftlichen Auseinandersetzungen führen.

